%\documentclass[cp, manuscript]{copernicus}
\documentclass[cp]{copernicus}

% Advances in Geosciences (adgeo)
% Advances in Radio Science (ars)
% Advances in Science and Research (asr)
% Advances in Statistical Climatology, Meteorology and Oceanography (ascmo)
% Annales Geophysicae (angeo)
% Atmospheric Chemistry and Physics (acp)
% Atmospheric Measurement Techniques (amt)
% Biogeosciences (bg)
% Climate of the Past (cp)
% Earth Surface Dynamics (esurf)
% Earth System Dynamics (esd)
% Earth System Science Data (essd)
% E&G Quaternary Science Journal (egqsj)
% Geoscientific Instrumentation, Methods and Data Systems (gi)
% Geoscientific Model Development (gmd)
% Hydrology and Earth System Sciences (hess)
% Journal of Sensors and Sensor Systems (jsss)
% Natural Hazards and Earth System Sciences (nhess)
% Nonlinear Processes in Geophysics (npg)
% Proceedings of the International Association of Hydrological Sciences (piahs)
% Scientific Drilling (sd)
% Solid Earth (se)
% The Cryosphere (tc)


%% \usepackage commands included in the copernicus.cls:
%\usepackage[german, english]{babel}
%\usepackage{tabularx}
%\usepackage{cancel}
%\usepackage{multirow}
%\usepackage{supertabular}
%\usepackage{algorithmic}
%\usepackage{algorithm}
%\usepackage{amsthm}
%\usepackage{float}
%\usepackage{subfig}
%\usepackage{rotating}
\usepackage{amsmath,amsfonts,amssymb}
\usepackage{url,hyperref}

\begin{document}
\title{Borehole temperatures revealsurface conditions during the Last Glacial Cycle}

\Author[1,2]{Volker}{Rath}
\Author[3]{Ilmo}{Kukkonen}
\Author[4]{Jan}{Sundberg}
\Author[5]{Jens-Ove}{Naeslund}
\Author[6]{Yuriy}{Maystrenko}

\affil[1]{Dublin Institute for Advanced Studies, Dublin, Ireland }
\affil[2]{Universit\'e de Savoie Montblanc, ISTerre, Le Bourget du Lac, France}
\affil[3]{Geophysical Institute, Helsinki Iniversity, Helsinki, Finland}
\affil[4]{JK Innova AB, Vadstena, Sweden }
\affil[5]{Swedish Nuclear Fuel and Waste Management Company (SKB),Stockholm, Sweden}
\affil[6]{Geological Survey of Norway (NGU), Trondheim, Norway}



\runningtitle{Boreholes temperatures}

\runningauthor{Rath et al.}

\correspondence{Volker Rath (vrath@cp.dias.ie)}



\received{}
\pubdiscuss{} %% only important for two-stage journals
\revised{}
\accepted{}
\published{}

%% These dates will be inserted by Copernicus Publications during the typesetting process.


\firstpage{1}

\maketitle



\begin{abstract}
TEXT
\end{abstract}


\copyrightstatement{TEXT}


\introduction  

Recent measurements of borehole temperature profiles (BTPs) can be used to estimate past 
climate change from the ground surface temperature (GST). Changes of surface temperatures
diffuse into the sub­surface according to their period. While signals of high-frequency 
like the annual temperature wave only pene­trate to a depth of 20~m, we can find the maximum
signature of the last glacial maximum (LGM, $\approx$20~kyr BP) depths between 1000~m and 2000~m, 
depending on the effective thermal diffusivity of the subsurface. A general de­scription of this 
method can be found in the monograph of \citet{Bodri2007a}. Since the seminal publi­cation of 
\citet{Lachenbruch1986a} this method has been employed in many studies on global, regional, or local 
scale (see the recent review of González-Rouco et al 2009, and the numerous references given 
therein). Most of these studies were based on very shallow boreholes (<500~m), thus only allowing a 
characterization of at most the last 
few centuries. Though already one of the first studies already dealt with the signature of the last 
glacial cycle \citep{Hotchkiss1934a}, the analysis of deeper boreholes has been less common 
\citep{Demezhko2012a,Kukkonen2011a,Kukkonen2011b,Chouinard2009a,Majorowicz2008a,Rath2007a,
Mottaghy2006a,Clauser1995a}. This is partly because of the lack of appropriate boreholes, partly 
becvause the diffusion-like behavior of heat conduction implies that the resolution of GSTH 
reconstructions decreases fast with age \cite[e.g.][]{Demezhko2001a}. We expect dominant signatures 
from the temperature increase from the LGM to the Holocene, and younger periods. Earlier 
contributions will be much less prominent, only determining the integral pre-LGM behavior. 

\citet{Rath2012a} have pointed out that the last glacial cycle influences BTPs even at shallow 
depths, where its effect amounts to an additional heat flow density, which does not invalidate most 
of the earlier studies using this data, though increasing error \citep{Beltrami2011a, Rath2012a}. 
For boreholes deeper than 500~m but not resolving the LGM ($>$1500~m), the effects will be clearly 
visible. Knowing the effects of the ling-term GSTH is not only crucial in the paleoclimate-related 
interpretation, but also will improve the significant bias in the estimate of ``geothermal'' 
background heat­ flow \cite[e.g.][]{Westaway2013a,Majorowicz2011a,Slagstad2009a}. 

\begin{figure} 
\begin{center}
\includegraphics[width=0.99\columnwidth]{Figures/BoreholeMap.png} 
\caption{Boreholes mentioned in this study.}
\end{center}
\end{figure}

\section{Physical and mathematical background}

The task of deriving meaningful estimates of past climate changes, in particular GSTHs, from recent 
borehole temperature profiles (BTP) is nontrivial, because the problem is strongly ill-posed in the 
nomenclature of \citet{Hadamard1923a}. In a mathematical sense, the solutions to this problem may 
not exist, they may be non-unique, and possibly inherently instable. Detailed accounts of this 
concept, and the techniques necessary to solve this kind of problems can be found in 
\citep{Hansen1998a,Hansen2010a} or \citep{Aster2013a}. In practice, this implies that the observed 
data (in this case the temperature measurements) need to be combined with knowledge from other 
sources to render the problem tractable. For the boreholes studied here, this additional information 
(as far as they are not introduced as explicit parameters in the Bayesian sense) includes a set of 
measured petrophysical properties, assumptions on the forward model used for the simulations, and on 
the general character of the solutions. It follows directly, that in these studies, the 
goodness-of-fit cannot be the single criterion for the quality of the solution. 

Most current software for the estimation of GSTHs from borehole temperature profiles is based on a 
one-dimensional, conduction-only forward model.

\section{Borehole Data and General Setup}
\label{sec:setup}

We have tied to 

As already pointed out in the introduction, the boreholes used in this study were not drilled for 
purely scientifc reasons, with the notable exception of  OKU. In particular, this means that the 
data sets availble are very inhomogeneous. Data and other relevant information site have to be 
pre-processed in a way appropriate for the given site, and assumptions have to be made accordinly. 
For the reconstruction of GSTHs buy means of inverse methods the following tasks need to be 
fullfilled in practice:

\subsection{Definition of the forward model}
As a first step, this includes the definition of appropriate spatial and temporal meshes. In this 
study, the spatial mesh is generally equidistant to the maximal depth  of the borehole, which 
guarantees equal statistical weights to the data associated to mesh nodes. For larger depths, cell 
sizes increase logarithmically until reaching a maximum extent of 5000~m. The time discretization 
starts at 10~yrs~BP, inceasing logarithmically to 110~kyrs~BP, which is the approximate periodicity
of glacial cycles in the last 800~kyr. The necessity to resolve the relevent tmperature changes 
requires 300 to 600 cells for this interval. 

\subsection{Setup of the inverse problem.} process and/or interpolate temperatures to mesh nodes, 
and choose the parameter vector $\mathbf{m}$.

 Thus 

\section{Results}
\subsection{Outokumpu,Finland}
\subsection{Olkiluoto,Finland}
\subsection{Forsmark, Sweden}
\subsection{Laxemar, Sweden}
\subsection{Bronn, Norway}

\conclusions  %% \conclusions[modified heading if necessary]
TEXT

%% The following commands are for the statements about the availability of data sets and/or software code corresponding to the manuscript.
%% It is strongly recommended to make use of these sections in case data sets and/or software code have been part of your research the article is based on.

\codedataavailability{TEXT} %% use this section when having data sets and software code available




\appendix

\section{Forward modelling}
\label{app:fwd}
The results presented in this article are based on one-dimensional, conduction-only forward 
modelling. Thermal conduction in rocks is based on the theory given by \citet{Carslaw1959a} for 
the heat conduction equation 

\begin{equation}\label{eqn:1}
 {\left( {\rho c} \right)_e}\frac{{\partial T}}{{\partial t}} + \frac{\partial 
}{{\partial z}}\left[ {{\lambda _e}\frac{{\partial T}}{{\partial z}}} \right] - 
H = 0{\text{ }.}
\end{equation} 

In this equation, $H$ denotes the volumetric heat production (W m$^{-3}$), while the index $e$ 
marks effective properties, i.e. properties which describe a rock-fluid-gas multi-phase system. As 
in the cases studied here porosity is very low, all properties will represent bulk rock properties, 
and the $e$ index will be omitted in the following. The occurring physical parameters are the 
density $\rho$ (kg m$^{-3}$), heat capacity $c_p$ (J kg$^{-1}$ K$^{-1}$), and thermal conductivity 
$\lambda$ (W m$^{-1}$ K$^{-1}$). The volumetric heat capacity is defined as $C = \rho c_p$. 
% diffusivity $\mu$ (m$^2$/s) 
% \begin{equation}\label{eqn:2}
% \mu = \frac{\lambda}{\rho c_p} = \frac{\lambda}{C}
% \end{equation} 

In order to solve Equation \ref{eqn:1} we assume a time-dependent Dirichlet boundary condition 
\begin{equation}\label{eqn:3a} 
T(z_0,t)=T_{GS}(t)
\end{equation} 
\noindent at the surface, and a stationary Neumann condition
\begin{equation}\label{eqn:3b}
\frac{\partial T}{\partial z}(z_{max }) =  - \frac{q_b}{\lambda_e}
\end{equation} 
\noindent at the base of the model domain. Here, $q_b$ is the basal heat flow density at the base 
 of the model, $z_{max}$. 
 
This forward problem is solved by numerical finite difference techniques 
\cite[e.g.,][]{Patankar1980a}, allowing for a flexible discretization in both, time and space. As 
shown in Figure~\ref{fig:FD}, temperatures in this finite difference (FD) scheme are associated 
to nodes located at the boundary of the cells. Thermal conductivity $\lambda$, density  $\rho$, 
heat capacity $c_p$, and heat production $H$ are associated with cells, but the latter is 
interpolated to the nodes for computational convenience. Time stepping is achieved by a Backward 
Euler or Crank-Nicholson scheme, which is complemented by a fixed point iteration for resolving the 
nonlinearities in the coefficients dicussed in Section~\ref{sec:setup}


\begin{figure}[htp]
 \centering
 \includegraphics[width=\columnwidth]{Figures/FDScheme}
 % FDScheme.png: 347x550 px, 220dpi, 4.01x6.35 cm, bb=0 0 114 180
 \caption[Schematic description of the finite difference discretization.]{Schematic description of 
the discretization used for 
the finite difference forward modeling. Measured data have to be assigned to 
computational nodes (red), while the properties are associated with the cells. 
Both may imply interpolations or averaging (upscaling) to a given grid for 
observed temperatures and measured petrophysical properties.}
\label{fig:FD}
\end{figure}



\section{Inversion techniques: Tikhonov}
\label{app:tikh}
For the deterministic approach we seek to minimize an objective function 
$\Theta$ which is defined by

\begin{equation}\label{eqn:4}
\Theta  = \mathcal{D}\left(\mathbf{d},\mathbf{m}\right) + \mathcal{R}\left(\mathbf{m} \right).
\end{equation} 
\noindent Here, the first term $\mathcal{D}$ measures the data fit, while the
second, $\mathcal{R}$ , is necessary to stabilize the generally ill-posed 
inverse problem, \cite[see][]{Aster2013a}. 

In this study, $\mathcal{D}$ is formulated as
\begin{equation} 
\mathcal{D} = 
 \left(\mathbf{d} - \mathbf{g(m)}\right)^T \mathbf{W}_d^T 
 \mathbf{W}_d^{}\left(\mathbf{d} - \mathbf{g(m)} \right)\quad,
\end{equation} 
\noindent where $\mathbf{W}_d$ denotes a weighting matrix commonly used to 
define the weighted residuals 
\begin{equation*}
\tilde{ \mathbf{r}} =\mathbf{W}_d^{}\mathbf{r} = 
\mathbf{W}_d^{} \left(\mathbf{d}-\mathbf{g}(\mathbf{m})\right)\quad, 
\end{equation*}

\noindent i.e., $\mathbf{W}_d = \mathbf{C}_d^{-1/2}$ is set to the inverse square root of the data 
covariance matrix $\mathbf{C}_d$, which assumed diagonal is in this study. 

In the problem treated here, the parameter vector $\mathbf{m}$ is composed of a piecewise constant 
function of time, using a logarithmic spacing. The data vector $\mathbf{d} = H(\mathbf{d}_{obs})$ 
contains the result of an observation operator $H$ applied to the discrete values of temperature 
measured in in the borehole. In this case, the application of this operator refers to the 
“upscaling” procedures mentioned in Section~\ref{sec:setup}. This regularization in 
Equation~\ref{eqn:4} can be achieved in many different ways. Most of the results presented here use 
a generalized Tikhonov technique, formulating the corresponding term as

\begin{equation}
\label{eqn:5}
\begin{gathered}
  \mathcal{R}\left( {{\mathbf{m}} - {{\mathbf{m}}_a}} \right) =  \hfill \\
  {\text{        }}\sum\nolimits_k {{\tau _k}} {\left( {{\mathbf{m}} - {{\mathbf{m}}_a}} 
\right)^T}{\mathbf{W}}_k^T{\mathbf{W}}_k^{}\left( {{\mathbf{m}} - {{\mathbf{m}}_a}} \right) \quad. 
\hfill \\ 
\end{gathered} 
\end{equation} 

In this equation, we define $\mathbf{W}_k$ to be a discrete approximation to the
first derivative of the parameters with respect to logarithmic time, 
\cite[see][]{Aster2013a}, which can be written in discrete form using a 
logarithmic time step $h$:
 	 
\begin{equation}\label{eqn:6}
{{\mathbf{W}}_1} = \frac{1}{h}\left[ {\begin{array}{*{20}{c}}
  { - 1}&1&{}& \cdots &{}&0 \\ 
  {}&{ - 1}&1&{}&{}&{} \\ 
   \vdots &{}& \ddots &{}&{}& \vdots  \\ 
  {}&{}&{}&{ - 1}&1&{} \\ 
  0&{}& \cdots &{}&{ - 1}&1 
\end{array}} \right]\end{equation} 

Taking the derivative of Equation~\ref{eqn:4}, and imposing the appropriate minimum conditions leads 
to the iteration
	
\begin{equation}\label{eqn:7}
\begin{gathered}
  \left(\mathbf{J}_w^T\mathbf{J}_w^{} + 
  \sum\limits_{i = 0}^1 \tau_i \mathbf{W}_{m,i}^T\mathbf{W}_{m,i}^{}\right)      
  \delta\mathbf{m}_k =  \hfill \\
  \quad \quad \quad \quad \mathbf{J}_w^T\left[\mathbf{d}-
  \mathbf{g}(\mathbf{m}_k) \right] - \\
  \quad \quad \quad \quad  \sum\limits_{i = 0}^1 \tau _i \mathbf{W}_{m,i} 
  \mathbf{W}_{m,i}^{}(\mathbf{m}_k - \mathbf{m}_a) \hfill \\ 
\end{gathered}
\end{equation} 

 \noindent with a subsequent update of the parameter vector by 
 $\mathbf{m}_{k + 1} = \mathbf{m}_k + \mu \delta\mathbf{m}_k$.

We define the weighted Jacobian or sensitivity matrix as

\begin{equation}\label{eqn:8}
\mathbf{J}_w \equiv \mathbf{W}_d\mathbf{J} = 
\mathbf{W}_d\partial{\mathbf{g}}/\partial {\mathbf{m}}\quad, 
\end{equation} 

and the Generalized Inverse as

\begin{equation}\label{eqn:9}
\mathbf{J}_w^G  = \left(\mathbf{J}_w^T \mathbf{J}_w^{} + 
\sum\limits_{i = 0}^1 \tau_i \mathbf{W}_{m,i}^T \mathbf{W}_{m,i}^{}\right)^{-1} 
\mathbf{J}_w^T{\text{  ,}}
\end{equation} 

respectively. Note that the inverse in Equation \ref{eqn:9} is the posterior covariance matrix. 
These quantities are appropriate tools to further analyze the sensitivity of given data to the 
inverse parameters. 

Given the two regularization parameters $\tau_0$ and $\tau_1$, this formulation can be seen an 
approximation to an inverse spatial exponential covariance a logarithmic scale with given error and 
correlation length \citep{Rodgers2000a, Tarantola2005a} For the inversions shown below, $\tau_0$ is 
set to a fixed small value (0.003), while $\tau_1$, is set to an optimal value determined by 
Generalized Cross Validation (GCV) \citep{Rath2007a,Farquharson2004a,Wahba1990a}. This optimal value 
is found by minimizing the GCV function 
\begin{equation}\label{eqn:10}
GCV(\tau ) = \frac
{N\left\| \mathbf{d} - \mathbf{g}(\mathbf{m}_\tau ^k) \right\|_2^2}
{\text{trace}
\left(\mathbf{I} - \mathbf{J}_w^{}\mathbf{J}_w^G \right)^2} 
\end{equation} 
\noindent over a set of predefined values of the regularization parameters $\tau$. For the 
inversions presented here, we always used the GCV criterion. The Unbiased Predictive Risk Estimator 
(UPRE) criterion described in \citet{Vogel2002a} produced nearly identical results. We did not use 
the L-curve approach described above, as this technique is not well adapted to non-linear inverse 
problems, where the identification of the point of maximal curvature often leads to highly 
improbable results. 

A useful measure for comparing the goodness-of-fit for different models is the normalized root mean 
square deviation (RMS) defined by
 
\begin{equation}\label{eqn:11}
RMS = \sqrt {\left( {{N_d} - 1} \right)_{}^{ - 1}\sum\nolimits_{j = 1}^N 
{{{\left( {\frac{{T_j^{obs}--T_j^{cal}}}{{{\sigma _j}}}} \right)}^2}} } \quad ,
\end{equation} 

where Nd is the number of observations, and the second term under the root is the sum of 
standardized residuals. In the case of $N_d$ being large, and approximately Gaussian and independent 
data, the value should be near one. However, due to the many assumptions involved, small differences 
in RMS are not considered as significant, and this parameter should only be used as an indicator. In 
particular, all runs shown here were run with a uniform error of 0.1~K for the temperature, which is 
usually a reasonable estimate for this quantity. Note that this error includes not only the 
measurement error, which is at least one order of magnitude smaller, but also some of errors 
related to the misspecification of the model (e.g., the use of average properties or interpolation).

\section{Numerical techniques: Markov Chain Monte Carlo}
In contrast to the deterministic approach described above, the Bayesian paradigm \cite[see, 
e.g.][]{Gelman2013a} aims not at a single GSTH which is optimal in a previously defined sense, but 
seeks to estimate the full posterior probability density function (PDF), given the prior PDF and the 
observational data. Detailed accounts on this approach can be found in 
\citep{Mosegaard1995a,Mosegaard2002a,Tarantola2005a}. In the present study we concentrate on the 
stochastic inversions. Deterministic inversions as described in Section 3.2.1 aim at deriving a 
single optimum model, which will commonly not catch the characteristics of all models compatible 
with the data. In contrast, the stochastic methods described in this section introduce constraints 
in generating a large number of models from a random set of temporally correlated models, as 
described below. Starting from the prior distribution by p(m), and the conditional probability 
distribution, p (d|m), also called likelihood, which describes the probability that the data, d, 
will be observed, given a set of parameters m. Given the above prior, we then seek the conditional 
(posterior) distribution of the model parameter(s) given the data. We will denote this posterior 
probability distribution for the model parameters by p(m|d). Bayes’ theorem relates prior and 
posterior distributions by
 
\begin{equation}\label{eqn:12}
p\left( {{\mathbf{m}}\left| {\mathbf{d}} \right.} \right) \propto p\left( 
{{\mathbf{d}}\left| {\mathbf{m}} \right.} \right)p\left( {\mathbf{m}} \right)
\end{equation} 

where the proportionality constant usually is not explicitly computed. While it is well known that, 
under the restrictive assumption of Gaussian probabilities, estimators can be derived, which bear 
some similarity to the deterministic methods described above \citep{Tarantola1982a,Tarantola1982b}, 
the technique of choice for the Bayesian is of stochastic nature. In this study, a Markov Chain 
Monte Carlo (MCMC) technique was employed. We used a variant of the well-known Metropolis-Hastings 
algorithm (MH) \citep{Metropolis1953a, Hastings1970a,Haario2006a}. The Delayed Rejection Adaptive 
Monte Carlo (DRAM) algorithm improves on MH, i) because it samples the posterior more effectively, 
as the rejection of low-likelihood samples is delayed for a predefined number of steps in the MC 
chain, before possibly being finally excludes, and ii) because it allows for an adaption of the 
prior covariance during the progress of the MC chain. We used the DRAM algorithm as implemented by 
\citet{Haario2006a}, with reduced or deactivated adaptivity in order to prevent a premature 
concentration of models. There are problems with this choice in the case of an ill-posed inverse 
problem as the one investigated here. The meaningfulness of the results depends strongly on the 
reasonable choice of both prior probability density, which was chosen to be a multidimensional 
Gaussian distribution $\mathcal{N}(mathbf{\mu},\mathbf{S})$. In DRAM, a predefined covariance matrix 
$\mathbf{C}$ is updated at given intervals during the MC process. In our case, we did not 
assume $\mathbf{C}$ to be diagonal, but allowed for temporal correlations in the GSTH parameters, 
while for the remaining parameters ($Q_b$ and $H$) it was initialised with the appropriate 
variances. For the parameters representing the GSTH, we introduced a Gaussian temporal covariance 
matrix defined as

\begin{equation}\label{eqn:13}
C_{i,j}^{{\text{gauss}}} = \sigma _i^2\exp \left[ {\frac{{ - {{\left| {{t_i} - 
{t_j}} \right|}^2}}}{{2{L^2}}}} \right] \quad .
\end{equation} 
 	f
Its width is controlled by a temporal correlation length $L$  of approximately 
half of a decade, which agrees with the resolution obtainable in this type of reconstruction (e.g. 
Demezhko and Shchapov 2001). A number of chains were started in parallel, each starting from a 
perturbed version of the prior which was set to the average of a GSTH derived from a Tikhonov 
deterministic inversion of the data set.
% 
% Figure 9‑31. Gaussian temporal co­var­ian­ces. In the MCMC samp­lings presented 
% in this study, correlation lengths of 5·Δt or less were used. 


\noappendix       %% use this to mark the end of the appendix section

%% Regarding figures and tables in appendices, the following two options are possible depending on your general handling of figures and tables in the manuscript environment:

%% Option 1: If you sorted all figures and tables into the sections of the text, please also sort the appendix figures and appendix tables into the respective appendix sections.
%% They will be correctly named automatically.

%% Option 2: If you put all figures after the reference list, please insert appendix tables and figures after the normal tables and figures.
%% To rename them correctly to A1, A2, etc., please add the following commands in front of them:

\appendixfigures  %% needs to be added in front of appendix figures

\appendixtables   %% needs to be added in front of appendix tables

%% Please add \clearpage between each table and/or figure. Further guidelines on figures and tables can be found below.



\authorcontribution{
VR developed the methodology and wrote the MATLAB code used in this study. The setup and discussion 
of each site was done by the respective groups: VR/IK for OKU and OLK, VR/JS/JN for 
FMK and LAX, and VR/JM for ULL. Laboratory measuremnts were done by IK and JS. All authors tool 
active part in the discussion of the results.   
} 

\competinginterests{
The authors declare that the research was conducted in the absence of any commercial or financial 
relationships that could be construed as a potential conflict of interest.
} 

\disclaimer{SKB, Posiva?} %% optional section

\begin{acknowledgements}
We are grateful to Heikki Hario, Markko Laine, and Aslak Grinsted for makeing their MCMC software 
available. Posiva Oi (OLkiluoto), SKB (Stockholm), and NGU allowed us to use their data for this 
study. Outolumpu was drilled as part of ICDP project XXXXX.  
\end{acknowledgements}




%% REFERENCES

%% The reference list is compiled as follows:

\bibliographystyle{copernicus} 
\bibliography{Refs}
%% Since the Copernicus LaTeX package includes the BibTeX style file copernicus.bst,
%% authors experienced with BibTeX only have to include the following two lines:
%%
%% \bibliographystyle{copernicus}
%% \bibliography{example.bib}
%%
%% URLs and DOIs can be entered in your BibTeX file as:
%%
%% URL = {http://www.xyz.org/~jones/idx_g.htm}
%% DOI = {10.5194/xyz}


%% LITERATURE CITATIONS
%%
%% command                        & example result
%% \citet{jones90}|               & Jones et al. (1990)
%% \citep{jones90}|               & (Jones et al., 1990)
%% \citep{jones90,jones93}|       & (Jones et al., 1990, 1993)
%% \citep[p.~32]{jones90}|        & (Jones et al., 1990, p.~32)
%% \citep[e.g.,][]{jones90}|      & (e.g., Jones et al., 1990)
%% \citep[e.g.,][p.~32]{jones90}| & (e.g., Jones et al., 1990, p.~32)
%% \citeauthor{jones90}|          & Jones et al.
%% \citeyear{jones90}|            & 1990



%% FIGURES

%% When figures and tables are placed at the end of the MS (article in one-column style), please add \clearpage
%% between bibliography and first table and/or figure as well as between each table and/or figure.


%% ONE-COLUMN FIGURES

%%f
%\begin{figure}[t]
%\includegraphics[width=8.3cm]{FILE NAME}
%\caption{TEXT}
%\end{figure}
%
%%% TWO-COLUMN FIGURES
%
%%f
%\begin{figure*}[t]
%\includegraphics[width=12cm]{FILE NAME}
%\caption{TEXT}
%\end{figure*}
%
%
%%% TABLES
%%%
%%% The different columns must be seperated with a & command and should
%%% end with \\ to identify the column brake.
%
%%% ONE-COLUMN TABLE
%
%%t
%\begin{table}[t]
%\caption{TEXT}
%\begin{tabular}{column = lcr}
%\tophline
%
%\middlehline
%
%\bottomhline
%\end{tabular}
%\belowtable{} % Table Footnotes
%\end{table}
%
%%% TWO-COLUMN TABLE
%
%%t
%\begin{table*}[t]
%\caption{TEXT}
%\begin{tabular}{column = lcr}
%\tophline
%
%\middlehline
%
%\bottomhline
%\end{tabular}
%\belowtable{} % Table Footnotes
%\end{table*}
%
%%% LANDSCAPE TABLE
%
%%t
%\begin{sidewaystable*}[t]
%\caption{TEXT}
%\begin{tabular}{column = lcr}
%\tophline
%
%\middlehline
%
%\bottomhline
%\end{tabular}
%\belowtable{} % Table Footnotes
%\end{sidewaystable*}
%
%
%%% MATHEMATICAL EXPRESSIONS
%
%%% All papers typeset by Copernicus Publications follow the math typesetting regulations
%%% given by the IUPAC Green Book (IUPAC: Quantities, Units and Symbols in Physical Chemistry,
%%% 2nd Edn., Blackwell Science, available at: http://old.iupac.org/publications/books/gbook/green_book_2ed.pdf, 1993).
%%%
%%% Physical quantities/variables are typeset in italic font (t for time, T for Temperature)
%%% Indices which are not defined are typeset in italic font (x, y, z, a, b, c)
%%% Items/objects which are defined are typeset in roman font (Car A, Car B)
%%% Descriptions/specifications which are defined by itself are typeset in roman font (abs, rel, ref, tot, net, ice)
%%% Abbreviations from 2 letters are typeset in roman font (RH, LAI)
%%% Vectors are identified in bold italic font using \vec{x}
%%% Matrices are identified in bold roman font
%%% Multiplication signs are typeset using the LaTeX commands \times (for vector products, grids, and exponential notations) or \cdot
%%% The character * should not be applied as mutliplication sign
%
%
%%% EQUATIONS
%
%%% Single-row equation
%
%\begin{equation}
%
%\end{equation}
%
%%% Multiline equation
%
%\begin{align}
%& 3 + 5 = 8\\
%& 3 + 5 = 8\\
%& 3 + 5 = 8
%\end{align}
%
%
%%% MATRICES
%
%\begin{matrix}
%x & y & z\\
%x & y & z\\
%x & y & z\\
%\end{matrix}
%
%
%%% ALGORITHM
%
%\begin{algorithm}
%\caption{...}
%\label{a1}
%\begin{algorithmic}
%...
%\end{algorithmic}
%\end{algorithm}
%
%
%%% CHEMICAL FORMULAS AND REACTIONS
%
%%% For formulas embedded in the text, please use \chem{}
%
%%% The reaction environment creates labels including the letter R, i.e. (R1), (R2), etc.
%
%\begin{reaction}
%%% \rightarrow should be used for normal (one-way) chemical reactions
%%% \rightleftharpoons should be used for equilibria
%%% \leftrightarrow should be used for resonance structures
%\end{reaction}
%
%
%%% PHYSICAL UNITS
%%%
%%% Please use \unit{} and apply the exponential notation


\end{document}
